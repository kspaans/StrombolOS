\documentclass{article}
\usepackage[left=2cm,top=1cm,right=3cm,nohead,nofoot]{geometry}
\usepackage{graphicx}


\usepackage[T1]{fontenc}
\usepackage[scaled]{helvet}
%\renewcommand*\familydefault{\sfdefault} 


\newcommand{\HRule}{\rule{\linewidth}{0.5mm}}

\begin{document}







\begin{titlepage}
\phantom{.}
\vspace{50pt}
\begin{center} 
\includegraphics{gstrombo}
\vspace{15pt}\\
\HRule \\[0.4cm]
\renewcommand*\familydefault{\sfdefault} 

{\huge \bfseries 

{        
        
\renewcommand*\rmdefault{helvet}\normalfont\upshape\sffamily
Documention for the {\sc StrombolOS}\texttrademark \\
\vspace{10pt}
Real-Time Operating System}
\vspace{12pt}


}


\HRule \\[1.5cm]
\end{center}
\begin{minipage}{0.4\textwidth}
\begin{flushleft} \large
\emph{Authors:}\\
Jacob Parker\\
Kyle Spaans
\end{flushleft}
\end{minipage}
\end{titlepage}


\tableofcontents
\pagebreak









\setcounter{section}{-4}
\section{Administrivia}

\section{Sample Output}
\section{README}

This is from the README file in our project's \texttt{docs/} directory. This entire 
document is also available as \texttt{DOCS/overview.pdf}.

\begin{verbatim}
To launch the executable from RedBoot, type ``l ARM/StrombolOS/k1'' and type 
``g'' after it loads.

It is strongly suggested that the ts7200 be reset before running this kernel 
incase someone else's kernel trashed the system.
\end{verbatim}

\section{Kernel Description}
\subsection{General Overview}
Our kernel's main function is organized as follows:

\begin{verbatim}
bootstrap ()
create_first_user_task ()

while (there is a task to run) {
  req <- get current task's return value from the kernel
  req <- activate (active task, req)
  req = handle(req, task descriptor)
  store req in the active task's task descriptor
  schedule ()
}
\end{verbatim}

\texttt{activate} is our context handler. It will restore a users state and then
jump to user code. When a software interrupt is triggered by a system call the 
cpu will jump back into the code for activate. The user state is saved and the 
kernel's is restored. The value returned by activate is the opcode of the system
call the user called.\\

\texttt{handle} takes the system call opcode and the relevant information for 
this task and does the appropriate thing. We have implemented \texttt{Pass}, 
\texttt{Exit}, \texttt{Create}, \texttt{MyTid}, \texttt{MyParentTid} as 
described in the kernel specification document.\\

\texttt{schedule} will take the current task and make it's state {\sc Ready} 
(only if it's current state is {\sc Active}) and pick the next task to run. 
This is described in more detail in the following section.\\

\subsection{Scheduling}
The type of scheduling used is round robin with priority levels. Our kernel will 
always run the task which is both in the state {\sc Ready} and of the highest 
priority.\\

The task descriptors have a next pointer in them of type 
pointer-to-a-task-descriptor. In this way a group of tasks can be stored in a 
single-linked list.\\

The task management system uses a pair of arrays of pointers to task descriptors. 
The first array, \texttt{p}, points to the last activated task of a given priority. 
The second array,\texttt{head}, points to the most recently created task of a given 
priority. In this way we are able to have circular buffers.\\

The \texttt{schedule} function starts at priority 0. If at any point in any priority 
it finds a task in the {\sc Ready} state it sets the state to {\sc Active}, updates the 
\texttt{p} pointer for that priority (for round robin scheduling) and returns a pointer 
to this task. If it cannot find a task of priority 0 that is {\sc Ready} it then tries 
level 1 and so on until it has exhausted all levels. If there is no task available to 
run, \texttt{schedule} will return 0 and the main kernel loop will exit, causing the kernel 
to exit and redboot to take over.\\

There are 6 priority levels. The justification for each level is given below. The names 
are considered self explainatory. As things change we will adjust the number of priority 
levels accordingly.

\begin{verbatim}
enum PRIORITY {
     INTERRUPT    = 0,
     SYSCALL_HIGH = 1,
     SYSCALL_LOW  = 2,
     USER_HIGH    = 3,
     USER_LOW     = 4,
     IDLE         = 5,
};
\end{verbatim}

\end{document}
