\documentclass{article}
%\include{Jake's Stuff}

\begin{document}

\section{Administrivia}
Build our kernel with \texttt{./Build.sh}, all build files, including
\texttt{StrombolOS.elf} will be in \texttt{build/}. The files and executable can
be found in \texttt{/u/j3parker/cs452/cs452-kernel/t1/}
\begin{verbatim}
9313c3b8d3a2e1d6d6f1162555ea3392  ./bwio/bwio.c
5bba63afffefa69c48ca4ee9ae7fc4cd  ./kernel/kernel.c
1861ba11f47ffed0386f26d5b1427fef  ./kernel/switch.h
f1460ffe80756a10326ea1f543d56b66  ./kernel/syscalls/create.c
4c445abfc7bbc947523149aa4ff8db69  ./kernel/syscalls/exit.c
88de93ad80f82c0c19068db925aa9a7b  ./kernel/syscalls/ksyscall.h
1fc352c47457bcc3483baf673453ca47  ./kernel/syscalls/myparenttid.c
7aea0293ec3376f3b609444439e0ffd8  ./kernel/syscalls/mytid.c
02a046f81c0b15aae6672817ba9c8fd7  ./kernel/syscalls/pass.c
e6193ca65075c3dc851866dfb593b5d7  ./kernel/syscalls/receive.c
62cfc465c6f3a121e6a1fc7ab37611fc  ./kernel/syscalls/reply.c
ad583b8bfa980bcd78c2666c0d993662  ./kernel/syscalls/send.c
60da756ed07348e576848ad2db5b9103  ./kernel/syscalls/awaitevent.c
ca28262728ec68a324728334cde3d85e  ./kernel/tasks.c
cf322c58d2b0ffca7960adb1731fc034  ./kernel/tasks.h
427aed49b53f52ce6185ed277e1046cb  ./kernel/boot.c
a3f52f48e4f13f1f4f3111c86733b828  ./kernel/boot.h
63a9c591765b838b80d156ba4d343ea9  ./kernel/switch.s
d32dda3f6cd59b210c03d1ed8332c581  ./include/bwio.h
8fe4bcbf13f60d8dfc91ecbdc0605734  ./include/debug.h
396d47c513a8d1763ec8524ef7043df3  ./include/ts7200.h
40f4f3cd6e750c354c5e0669cc79afa0  ./include/ANSI.h
71fab5b6b9f20b06d3fee8a131e488ac  ./user/lib.c
f6a2380d93ce32a1588c9f7e8672de8b  ./user/lib.h
bb26439fa9e96b4ad416cca1ffbee9d8  ./user/rps_client.c
987391166fc3a7911abaade6e911228f  ./user/usyscall.s
5244b31fad05d2c4a13f3af00151d689  ./user/user.c
2737fde1654b841a2a8273a03121a808  ./user/user.h
26e6c3c23813ba23d358bf6cef18a89b  ./user/usyscall.h
100386ccd693a1b05231723e825a7056  ./user/wrappers.c
1fc7829e40ad9731b92880bd06e5ae6a  ./user/clock_client.c
0a4ecd61c17f8fab2701df0fc602f52b  ./user/clock_client.h
5cdf0685d1b04b08223b59fe346c2a2b  ./user/timings.c
a66c2ece6a834d354855b5a349577f16  ./user/trains_ui.c
fdb022424f137015c687e9233746011d  ./user/wm.c
a499e8d19ee5410832196b18d02604cc  ./ktests/tests.c
55ff8b30fa5d99a344fe21c4f04efa10  ./ktests/tests.h
7d2b05ec0779ca90badce6913bb8abe4  ./servers/names.c
543b0ddeef7d66f66b7126efa9b58ea0  ./servers/rand.c
53eb72a04263b25510535132b3453e35  ./servers/rps_server.c
40cfb1a75cc25c85ee53aeea72f1cffa  ./servers/servers.h
e7394261773af98b40e1363d90fa8b1e  ./servers/clock.c
1b884ba7141569e43c5e611cd5ca2b37  ./servers/notifier_clock.c
775f0fd0347788f42800f45778fa1208  ./servers/uart1.c
9b245bca020cc929b760e3d1e44318d4  ./servers/uart2.c
304d3518e6d323b70013aa816fc26a08  ./servers/notifier_uart1tx.c
1c31285b50f4d5d80396fb988ac672f9  ./servers/notifier_uart1rx.c
843683fbdf9836dbe8454cfee822b646  ./servers/notifier_uart2tx.c
73f601af47ba67c47b15ba051888c168  ./servers/notifier_uart2rx.c
6fc27c95fb0c4f9fc6c2e352f62ce1d3  ./servers/trains.c
343b974fd07f3cd492416b2ea92b7bc6  ./Build.sh
9e55df43bdbf8cb6b4743399b06df38b  /u/cs452/tftpboot/ARM/StrombolOS/k4
\end{verbatim}

\section{REAME}
Please run
\texttt{l ARM/StrombolOS/t1; go -c} on a freshly booted ARM board.
New commands are
\begin{itemize}
\item \texttt{trap s}, to have the active train's speed automatically set to
zero when it trips sensor $s$
\item \texttt{add n}, to add train $n$ to the list of active trains to be
tracked
\item \texttt{loc n}, to report the location (sensor) of train $n$
\end{itemize}
The recommended way to use our kernel is:
\begin{verbatim}
go
tr n 0
swall C
tr n s
add n
...
\end{verbatim}
To make sure that switches and trains are properly initialized.

\section{Kernel Description}

For datastructures we use arrays of values. Specific values such as train
speed, location, switch states, etc are passed back and forth by the train
infrastructure in messages. We don't need data structures that are more
elaborate because we can always translate an identifier (for a train, sensor, or
switch) into an index in constant time. The train's position is maintained as
the sensor ID and time the sensor was tripped. The next (expected, given
information about the track such as switch state which may not be accurate)
sensor can alsways be deterministically determined in a short amount of time
by traversing a graph representing the topology of the track. The current
position is estimated given the train's current speed and the time since it
tripped the sensor, a simple calculation.

\subsection{Calibration}
So far, we assume that the trains maintain a constant speed over every portion
of the track. We have measured the average speeds for trains 32 and 52, at three
selected speeds: 8, 10, 14. We have also measured the stopping distances for
each of these speeds. This distance is used to help estimate the final position
when the train is stopped (\texttt{tr $x$ 0}).
%Given stopping distance
%$\delta x$, and current velocity $v_0$, the intermediate position (between $t_0$
%and the stopping position) can be calculated with
%\[ \frac{(t - t_0)ddx}{dt} \]
The current speed is updated by incorporating the amount of time between the
last two successive sesnors that the train has tripped and the distance between
those sensors (as reported by the track server).

\subsection{Servers}
Need more here?

\end{document}
