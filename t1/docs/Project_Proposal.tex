\documentclass{article}
\usepackage{fullpage}

\title{CS 452 Project Proposal}
\author{Jacob Parker, Kyle Spaans}
\date{Monday November 29, 2010}

\begin{document}

\maketitle

\section*{In Which the Trains are Entrained}
Also known as, ``Entrain Tracking'' or perhaps ``$N$-Train Tracking'' if you're
so inclined.

\subsection*{Overview}
For our final project, we propose to track more than two trains. We plan to
track at least 3 trains on the track simultaneously. $N$ trains will be
travelling on the track, either in loops, or to and from destinations, or both.
The trains will throttle their speeds or stop to avoid collisions.

\subsection*{Technical Challenges}
\begin{itemize}
\item Tracking multiple trains requires tight calibration.
\item We need to finish the better lost-train algorithm as described in our T2
documentation. We have to be careful when trying to find ourselves because we
wish to avoid teleportation.
\item We will need to have better deadlock dedication and resolution (when
possible) strategies.
\end{itemize} 

\subsection*{Technical Solutions}
\begin{itemize}
\item For better calibration we will need to spend more time calibrating trains
instead of our current data (which was derived from one train and tweaked as
needed.)
\item The lost algorithm is partially written but has some bugs that need to be
ironed out. 
\item With three trains the possible combinations for deadlocking are more
severe. There now exist impossible cases such as one train being trapped in
between two stopped trains. We need to be able to detect this and display a
friendly message.
\item It may help our reservation code if we reduce the size of segments (i.e.
subdivide the track.) This has limited utility in tight sections of the track
(subdivided lines may not have a sensor) but on longer portions of the track it
may be necessary so as to prevent deadlocks.
\end{itemize}
\end{document}
